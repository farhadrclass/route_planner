\section{Literature Review}
To achieve the objectives presented in this project pre-existing material on path planning, tour planning, energy routing and sampling plans has to be considered.

\cite{Dubins1957} and \cite{Boissonnat1993} consider the shortest path of a vehicle with a bounded turning radius in two dimensions: given a trajectory and location for the beginning and end points. Both papers found that the minimum path is comprised of maximum rate turns and straight line segments. This initial work on shortest paths is extended by \cite{Chitsaz2007} to take a third dimension into account: for low altitude ranges the shortest path is the Dubin path in the x-y plane and a constant rate altitude climb. High altitude climbs diverged from the Dubin path due to helical climbing component however this is not relevant as only low altitude climbs are considered in this project.

\cite{McGee2005} and \cite{Techy2009} apply Dubins shortest path in uniform wind. This is done by considering a ground reference frame and wind reference frame which results in the Dubin minimum path being calculated in the wind reference frame. The maximum rate turns in the air frame of reference "correspond to trochoidal paths in the inertial [ground] frame" \cite{Techy2009}. Assuming uniform and time-invariant wind leads to potential inaccuracies which are considered through using "a turning rate less than the actual maximum turning rate" \cite{McGee2005}. Both papers considered utilise different approaches to obtain a final optimal path, thought the resulting optimal paths are both comprised of a combination of trochoidal path sections and straight line sections. These papers are limited in there application to this paper as they only consider a set of two points in two dimensions.

\cite{Bigg1976}, \cite{Held1984} and \cite{DeBerg2010} calculate minimum length tours of more than two points through applying the travelling salesperson problem (TSP). The TSP is concerned with "finding the shortest path joining all of a finite set of points whose distances from each other are given" \cite{Held1984}. The TSP problem considered in these papers uses the euclidean distance where the "euclidean distances satisfy the triangle inequality" \cite{DeBerg2010}; "The triangle inequality implies that no reasonable salesman would ever revisit the same city: Instead of returning to a city, it is always cheaper to skip the city and to travel directly to the successor city" \cite{DeBerg2010}. Given the euclidean distances the solutions obtained do not correspond to the shortest path for a Dubin vehicle as the turning angle at nodes is not considered.

\cite{Savla2005a}, \cite{Savla2005} and \cite{LeNy2008} consider the Dubin travelling salesperson problem (DTSP). The basic approach considered in all papers is in the form of the alternating algorithm which requires calculation of a minimum tour using euclidean distances for an initial ordering. From the initial order the heading at nodes is given by the direction of either the vertex before the node or the vertex after the node. With the order, location and heading defined at each point the Dubin shortest path can be calculated. \cite{Savla2005} goes on to consider stochastic DTSP where the points are normally distributed and puts forth a bead tilling algorithm to improve the performance which is an important consideration for the initial planning aspect of this paper. \cite{LeNy2008} however goes beyond the scope of this paper in considering variable vehicle dynamics.

\cite{Al-Sabban}, \cite{Chakrabarty2009} and \cite{Langelaan2007} look into the minimum energy paths through non-uniform wind vectors by considering the total energy of the UAV and attempting to minimise the reduction in energy. \cite{Al-Sabban} uses a markov decision process to plan a route through time varying wind vectors which have a degree of uncertainty. \cite{Chakrabarty2009} and \cite{Langelaan2007} however use a predetermined knowledge of the wind with the aim to exploit atmospheric energies. The given equations of energy and considerations of optimal routes through complex wind fields applies directly to this paper however attempting to tap into soaring flight is not feasible given the limited research area.

\cite{Forrester2008} and \cite{McKay2000} consider efficient sampling plans for black box experiments to improve the quality of the model produced. Both papers present latin hypercube sampling to be an improvement on random sampling as they ensure "that each of those components is represented in a fully stratified manner" \cite{McKay2000} where those components refers to input dimensions. \cite{Forrester2008} extend this by optimising latin hypercubes to result in the plan with best space fillingness. The sampling plans provided can easily be utilised in the primary planning component of the UAV tour in this project.
