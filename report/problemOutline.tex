\section{Problem Outline}
The troposphere is the lowest atmospheric layer and of significant interest to meteorological researchers as "almost all weather develops in the troposphere" \cite{NatGeo}. The section of the troposphere that is closest to the earth is the atmospheric boundary layer, in this layer the atmospheric conditions are affected by the surface of the earth. These affects mean that modelling this section is much more complicated than other layers of the atmosphere, therefore "the boundary layer is still not represented realistically" \cite{Teixeira2008}. Developing a greater understanding and thus better modelling of the atmospheric boundary layer will only improve the ability to: correctly predict weather forecasts or pollutant dispersion, to name but a few. 

To collect atmospheric data a number of approaches are available, "many years before the use of radio-controlled aircraft, the collection of in-situ measurements was primarily done with balloons, towers, and tethersondes" \cite{Bonin2011}. These options are limited as they are not able to move within the area of interest to build a model. UAVs operate with "reduced human risk, but also reduced weight and cost, increased endurance, and a vehicle design not limited by human physiology" \cite{Pepper2012} in comparison to their manned counterparts. This means they provide a cheap and mobile data collection platform.

There are 5 main classes of UAV for different requirements \cite{Sarris2001}  the class of UAV best suited to collecting atmospheric data of a limited area is the close range class. This class of UAV: "require minimum manpower, training, and logistics, and will be relatively inexpensive" \cite{FasDod}. These benefits allow a greater number of researchers to have access to mobile data collecting platforms. Most small UAVs are "not capable of reaching above 5,000ft [1524m]" \cite{Weibel2005} with their standard operating level generally not exceeding 350m and their maximum range being less than 10km \cite{Blyenburgh2000}.
%See appendix \ref{sec:uav_information} for more information on UAV classes.

The ideal solution to the problem of collecting atmospheric data will take a form where the limited energy contained within the UAV is maximally utilised to ensure the best possible sampling of the area of interest. In addition due to the atmospheric irregularities present in the lower boundary layer the energy consideration of the planning element will have to utilise a non-uniform vector field. 
